\documentclass[11pt]{article}
\usepackage[utf8]{inputenc}

\usepackage{latexsym}
\usepackage{amssymb,amsmath}
\usepackage{graphicx}
\usepackage{sgame}
\usepackage{xcolor}
\usepackage{authblk}

%\usepackage{indentfirst}
\usepackage[toc,page]{appendix}
\renewcommand{\appendixname}{Appendix}
\renewcommand{\appendixtocname}{Appendix}
\renewcommand{\appendixpagename}{Appendix}

\usepackage[hidelinks]{hyperref}
\usepackage{empheq}
\usepackage{blkarray}
\usepackage{cancel}
\usepackage{enumerate}
\usepackage{times}
\usepackage{array}
\usepackage{lscape}
\usepackage{setspace}

\usepackage[margin=1in]{geometry}
\newcommand{\newword}[1]{\textbf{\emph{#1}}}

%Arrows
\newcommand{\into}{\hookrightarrow}
\newcommand{\onto}{\twoheadrightarrow}

%Macros
\newcommand{\isom}{\cong} %The isomorphism symbol
\newcommand{\union}{\cup}
\newcommand{\intersection}{\cap}
\newcommand{\bigunion}{\bigcup}
\newcommand{\bigintersection}{\bigcap}
\newcommand{\disjointunion}{\sqcup}
\newcommand{\bigdisjointunion}{\bigsqcup}

\newcommand\numberthis{\addtocounter{equation}{1}\tag{\theequation}}

%Some multiletter functions
\DeclareMathOperator{\Hom}{Hom}
\DeclareMathOperator{\Ext}{Ext}
\DeclareMathOperator{\End}{End}
\DeclareMathOperator{\Tor}{Tor}
\DeclareMathOperator{\Ker}{Ker}
\DeclareMathOperator{\CoKer}{CoKer}
\DeclareMathOperator{\Spec}{Spec}
\DeclareMathOperator{\Proj}{Proj}
\renewcommand{\Im}{\mathop{\mathrm{Im}}}
%Their calligraphic versions; use these for the sheaf constructions
\DeclareMathOperator{\HHom}{\mathcal{H} \textit{om}}
\DeclareMathOperator{\EExt}{\mathcal{E} \textit{xt}}
\DeclareMathOperator{\EEnd}{\mathcal{E} \textit{nd}}
\DeclareMathOperator{\TTor}{\mathcal{T} \textit{or}}
\DeclareMathOperator{\KKer}{\mathcal{K}\textit{er}}
\DeclareMathOperator{\CCoKer}{\mathcal{C} \textit{o}\mathcal{K} \textit{er}}
\newcommand{\IIm}{\mathop{\mathcal{I} \textit{m}}}
\newcommand{\ccH}{\mathscr{H}} %The very curly H

\DeclareMathOperator{\sss}{\mathrm{sunny}}
\DeclareMathOperator{\rrr}{\mathrm{rainy}}
\DeclareMathOperator{\hhh}{\mathrm{hot}}
\DeclareMathOperator{\ccc}{\mathrm{cold}}


%This makes alternating tensors look right in displayed equations
\newcommand{\Alt}{\bigwedge\nolimits}

%Blackboard bold letters

\renewcommand{\AA}{\mathbb{A}}
\newcommand{\BB}{\mathbb{B}}
\newcommand{\CC}{\mathbb{C}}
\newcommand{\DD}{\mathbb{D}}
\newcommand{\EE}{\mathbb{E}}
\newcommand{\FF}{\mathbb{F}}
\newcommand{\GG}{\mathbb{G}}
\newcommand{\HH}{\mathbb{H}}
\newcommand{\II}{\mathbb{I}}
\newcommand{\JJ}{\mathbb{J}}
\newcommand{\KK}{\mathbb{K}}
\newcommand{\LL}{\mathbb{L}}
\newcommand{\MM}{\mathbb{M}}
\newcommand{\NN}{\mathbb{N}}
\newcommand{\OO}{\mathbb{O}}
\newcommand{\PP}{\mathbb{P}}
\newcommand{\QQ}{\mathbb{Q}}
\newcommand{\RR}{\mathbb{R}}
\renewcommand{\SS}{\mathbb{S}}
\newcommand{\TT}{\mathbb{T}}
\newcommand{\UU}{\mathbb{U}}
\newcommand{\VV}{\mathbb{V}}
\newcommand{\WW}{\mathbb{W}}
\newcommand{\XX}{\mathbb{X}}
\newcommand{\YY}{\mathbb{Y}}
\newcommand{\ZZ}{\mathbb{Z}}

%Calligraphic letters

\newcommand{\cA}{\mathcal{A}}
\newcommand{\cB}{\mathcal{B}}
\newcommand{\cC}{\mathcal{C}}
\newcommand{\cD}{\mathcal{D}}
\newcommand{\cE}{\mathcal{E}}
\newcommand{\cF}{\mathcal{F}}
\newcommand{\cG}{\mathcal{G}}
\newcommand{\cH}{\mathcal{H}}
\newcommand{\cI}{\mathcal{I}}
\newcommand{\cJ}{\mathcal{J}}
\newcommand{\cK}{\mathcal{K}}
\newcommand{\cL}{\mathcal{L}}
\newcommand{\cM}{\mathcal{M}}
\newcommand{\cN}{\mathcal{N}}
\newcommand{\cO}{\mathcal{O}}
\newcommand{\cP}{\mathcal{P}}
\newcommand{\cQ}{\mathcal{Q}}
\newcommand{\cR}{\mathcal{R}}
\newcommand{\cS}{\mathcal{S}}
\newcommand{\cT}{\mathcal{T}}
\newcommand{\cU}{\mathcal{U}}
\newcommand{\cV}{\mathcal{V}}
\newcommand{\cW}{\mathcal{W}}
\newcommand{\cX}{\mathcal{X}}
\newcommand{\cY}{\mathcal{Y}}
\newcommand{\cZ}{\mathcal{Z}}


\DeclareMathOperator{\ord}{ord}
\DeclareMathOperator{\inte}{int}
\DeclareMathOperator{\nhd}{nhd}

\newcommand{\ds}{\displaystyle}
\newcommand{\mc}{\mathcal}
\newcommand{\ol}{\overline}
\newcommand{\modu}{\hspace{-2mm} \mod}

\DeclareMathOperator{\inn}{Inn}
\DeclareMathOperator{\aut}{Aut}
\DeclareMathOperator{\cen}{Center}
\DeclareMathOperator{\im}{Im}
\DeclareMathOperator{\re}{Re}
\DeclareMathOperator{\id}{id}
\DeclareMathOperator{\mor}{Mor}
\DeclareMathOperator{\irr}{Irr}
\DeclareMathOperator{\sgn}{sgn}

\DeclareMathOperator{\cov}{Cov}
\DeclareMathOperator{\var}{Var}

\DeclareMathOperator{\erf}{erf}
%\DeclareMathOperator{\sgn}{sgn}
\DeclareMathOperator{\argmin}{argmin}
\DeclareMathOperator{\argmax}{argmax}

\DeclareMathOperator{\lip}{Lip}

\newcommand{\bbm}{\begin{bmatrix}}
\newcommand{\bpm}{\begin{pmatrix}}
\newcommand{\ebm}{\end{bmatrix}}
\newcommand{\epm}{\end{pmatrix}}

\newcommand{\ddx}[2]{\frac{d #1}{d #2}}
\newcommand{\ddt}[1]{\frac{d #1}{dt}}

 \newcommand{\del}[2]{\frac{\partial #1}{\partial #2}}
 \newcommand{\dsdel}[2]{\displaystyle\frac{\partial #1}{\partial #2}}
 
 \newcommand{\doubledel}[3]{\displaystyle\frac{\partial^2 #1}{\partial #2 \partial #3}}
 \newcommand{\doubledelsame}[2]{\displaystyle\frac{\partial^2 #1}{\partial #2^2}}
  
%newcommand{\ddx}[2]{\frac{d #1}{d #2}}
%\newcommand{\ddt}[1]{\frac{d #1}{dt}}

\newcommand{\dsddx}[2]{\displaystyle\frac{d #1}{d #2}}
\newcommand{\dsddt}[1]{\displaystyle\frac{d #1}{dt}}

\newcommand{\pbderiv}{\ds\del{V}{x_1} \dsddt{x_1} + \ds\del{V}{x_2} \dsddt{x_2}}

\newcommand{\ito}{It\^o \hspace{0.05mm}}
\newcommand{\itos}{It\^os \hspace{0.05mm}}

\newcommand{\gronwall}{Gr\"onwall  \hspace{0.05mm}}
\newcommand{\gronwalls}{Gr\"onwall's  \hspace{0.05mm}}

\newcommand{\tw}{d\tilde{W}_t}
\newcommand{\tws}{d\tilde{W}_s}

\bibliographystyle{plain}
\usepackage{float}

\newcommand{\A}{{\color{red}A}}
\newcommand{\B}{{\color{blue}B}}
\newcommand{\later}{{\color{red}(Add later)}}


%%%%%%%%%%%%%%%%%%%%%%%%%%%
% Document-specific settings

\title{\vspace{-30pt}Updates on the Fixed Threshold Mixing Model}
\author{Mari Kawakatsu and Christopher K. Tokita\vspace{-10pt}}
\date{Last updated: \today}

\graphicspath{ {../output/Task_dist/} }
\usepackage[labelfont=bf,margin=.3in]{caption}

%%%%%%%%%%%%%%%%%%%%%%%%%%%
\begin{document}

\maketitle
% \noindent
% {\color{red}I am thinking that we can make this an internal document for our records, since I think most of the theoretical results we have are useful for our understanding but don't need to be included in whatever we share with Yuko et al. We can make a separate document for them with key points and questions.}

\tableofcontents

\section{Overview}

\newpage
\section{Varying both the task demand rate and the task efficiency (Mari)} \label{sec:varyalphadelta}

To capture the effects of the different larvae on task performance, we varied the task demand rate ($\delta$) in addition to the task performance efficiency ($\alpha$). In particular, we were interested in whether this combination could give rise to the different directions of contagion (upward or downward) observed in the data. 
For simplicity, these simulations assumed that the tasks associated with a given type of larvae (\A\ or \B) have the same demand rate.

When both lines are sufficiently efficient, changing the demand rate makes no qualitative difference in the relative levels of the mean colony activity in the different mixes (pure \A, pure \B, and 50-50 mixes of \A\ and \B). Within each mix, the more demanding the tasks (higher $\delta$), the higher the mean activity (Fig.~\ref{fig:deltassuperefficient}).  Moreover, the difference in the mean activity levels for pure-\A\ and pure-\B\ colonies (i.e., the gap between the pure-\A\ and pure-\B\ means) is more pronounced for the more demanding tasks. Despite these quantitative differences, the qualitative picture remains unchanged:
we observe a downward contagion \textit{both} when the tasks are less demanding ($\delta = 0.6$, Fig.~\ref{fig:deltassuperefficient}\textbf{A}) and more demanding ($\delta = 0.8$, Fig.~\ref{fig:deltassuperefficient}\textbf{B}).

\begin{figure}[H]
    \centering
    \includegraphics[trim={0 1.1in 0 1.1in}, clip, width=0.95\linewidth]{{doc/deltas_comparison_superefficient}.pdf}
    \caption{Simulation results for varying both the demand rate ($\delta$) line \B\ is highly efficient ($\alpha_{j}^{\B} = 6$). \textbf{A}:~Tasks are less demanding ($\delta = 0.6$). \textbf{B}:~Tasks are more demanding ($\delta = 0.8$). Parameters: $\alpha_{j}^{\A}  = 2$, $\sigma = 0.1$, $\mu = 10$, $\eta = 7$.}
    \label{fig:deltassuperefficient}
\end{figure}

However, when one line is inefficient, changing the demand rate changes the picture qualitatively. When the tasks are less demanding ($\delta = 0.6$), we observe an upward contagion (Fig.~\ref{fig:deltasinefficient}\textbf{A}). When the tasks are more demanding ($\delta = 0.8$), however, the task performance levels become nearly indistinguishable among the three mixes (Fig.~\ref{fig:deltasinefficient}\textbf{B}). We think that this qualitative changes appears because the less efficient ants are unable to keep the stimuli at a steady level: the task performance level for the colony consisting entirely of the inefficient line (pure~\B) remains unchanged for $\delta = 0.6$ and $\delta = 0.8$, which suggests that the ants are working at maximum capacity. When we check the simulation results, we observe that indeed both stimuli keep increasing over time (i.e., do not reach a steady state).

\begin{figure}[H]
    \centering
    \includegraphics[trim={0 1.1in 0 1.1in}, clip, width=0.95\linewidth]{{doc/deltas_comparison_inefficient}.pdf}
    \caption{Simulation results for varying both the demand rate ($\delta$) line \B\ is inefficient ($\alpha_{j}^{\B} = 1$). \textbf{A}:~Tasks are less demanding ($\delta = 0.6$). \textbf{B}:~Tasks are more demanding ($\delta = 0.8$). Parameters: $\alpha_{j}^{\A}  = 2$, $\sigma = 0.1$, $\mu = 10$, $\eta = 7$.}
    \label{fig:deltasinefficient}
\end{figure}

The above observations are based on the case in which one line is more efficient at both tasks than the other line. We also tested the case in which each line is more efficient at one task but less efficient at the other task (e.g., $\alpha_{1}^{\A} = \alpha_{2}^{\B} = 2, \alpha_{1}^{\B} = \alpha_{2}^{\A} = 6$ or $1$)\footnote{We would be happy to discuss these results if you are interested.}. Taken together, the results suggest that \textbf{the demand rate has \textit{no qualitative effect} when all ants are sufficiently efficient but \textit{a nontrivial effect} when some ants are inefficient.}

% \begin{figure}[H]
%     \centering
%     \includegraphics[trim={0 0.4in 0 0.2in}, clip, width=0.9\linewidth]{{doc/deltas_0.6_0.6_comparison}.pdf}
%     \caption{Caption}
%     \label{fig:deltas0606}
% \end{figure}

% \begin{figure}[H]
%     \centering
%     \includegraphics[trim={0 0.4in 0 0}, clip, width=0.9\linewidth]{{doc/deltas_1.0_1.0_comparison}.pdf}
%     \caption{Caption}
%     \label{fig:deltas1010}
% \end{figure}

% \begin{figure}[H]
%     \centering
%     \includegraphics[trim={0 0 0 0}, clip, width=0.9\linewidth]{{doc/deltas_0.6_1.0_comparison}.pdf}
%     \caption{Caption}
%     \label{fig:deltas0610}
% \end{figure}

\newpage
\section{Varying ratios of the genetic lines (Chris)}
Increasing the number of mix ratios demonstrates a non-linear effect with regard to task performance. When exploring these mixes, we held genetic line A as "average" in task efficiency (i.e., $\alpha_j^A = 2$) and varied the task efficiency of genetic line B. For line B, we assumed either high efficiency (i.e., $\alpha_j^B = 6$) or inefficiency (i.e., $\alpha_j^B = 1$). When B is highly efficient the frequency of task 1 performance increases exponentially as the proportion line B increases. On the other hand, when B is inefficient task 1 performance decreases non-linearly as the proportion of B increases. Overall, this predicts that experiments with more mixing ratios will show a nonlinear trend, assuming that task efficiency is the main difference between the two lines. 

\begin{figure}[H]
    \centering
    \includegraphics[trim={0 0.25in 0 0.2in}, clip, width=0.9\linewidth]{doc/Mix_Alphas_B-super-efficient_Means.png}
    \caption{Simulation results for mixing of genetic lines when line B is highly efficient. Parameters: $\alpha_j^A = 2, \alpha_j^B = 6,$}
    \label{fig:Mix_Alphas_B-efficient}
\end{figure}

\begin{figure}[H]
    \centering
    \includegraphics[trim={0 0.25in 0 0.2in}, clip, width=0.9\linewidth]{Mix_Alphas_B-inefficient_Means.png}
    \caption{Simulation results for mixing of genetic lines when line B is inefficient. Parameters: $\alpha_j^A = 2, \alpha_j^B = 1,$}
    \label{fig:Mix_Alphas_B-inefficient}
\end{figure}

\section{Analytical model (Mari)}
% \begin{itemize}
%     \item This is the model
%     \item Here are the predictions that the model makes
%     \item This is what we would like you to test
% \end{itemize}

To better understand our simulation results, we have developed and conducted a preliminary analysis of an analytical model with two different types of ants.  The details of the model and the analysis are outlined in the \hyperref[sec:appendix]{Appendix}; this section summarizes the most important takeaways.

\textbf{Model summary}. The model considers $n$ individuals and $m$ tasks. The fractions of line \A\ and \B\ individuals are given by $f$ and $1-f$, respectively. For example, $f=1$ corresponds to a pure \A\ colony, $f = 0.5$ to a 50-50 mixed colony, and $f = 0$ to a pure \B\ colony. 
% As before, $\delta_j$ is the task-specific demand rate, and $\alpha_j^{\A}$ and $\alpha_j^{\B}$ are the task-specific performance efficiencies of lines \A\ and \B, respectively.
The model captures the dynamics of 1) the number of line \A\ and line \B\ individuals performing task $j$ at a given time, denoted $n_{j,t}^{\A}$ and $n_{j,t}^{\B}$ respectively, as well as 2) the stimuli associated with the two tasks. See Appendix~\ref{sec:model} for details.

\textbf{Prediction 1}: The ants must be sufficiently efficient in order for the system to reach a steady state (See Section~\ref{sec:maxactivity}). Specifically, the model parameters must satisfy the following conditon \later

\textbf{Prediction 2}: Steady state values

\textbf{Prediction 3}: Non-50-50 mixes

\textbf{Prediction 4}: No downward contagion possible for equal $\mu$ and $\tau$.


Upward contagion is still possible if we relax some of our assumptions. In the simulations I've run so far, upward contagion appears
\begin{itemize}
    \item when the mean thresholds ($\mu$) are varied in addition to the task efficiencies ($\alpha$), or
    \item when at least one of the lines is not sufficiently efficient to maintain the stimuli at constant levels (i.e., reach steady state).
\end{itemize}

\begin{figure}[H]
    \centering
    \includegraphics[trim={0 0.25in 0 0.2in}, clip, width=0.95\linewidth]{5050_comparison.pdf}
    \caption{Sim results + predictions need to be separated}
    \label{fig:5050comp}
\end{figure}

\begin{figure}[H]
    \centering
    \includegraphics[trim={0 1in 0 1.1in},clip,width=0.95\linewidth]{mixes_comparison.pdf}
    \caption{Yay}
    \label{fig:mixescomp}
\end{figure}
\footnote{How picky are we about matching font sizes?}

\section{Conclusions \& Questions} \label{sec:conclusions}

Questions for them:
\begin{itemize}
    \item Which line is dominant? In other words, which line is more likely to survive long-term?
    \item Are both lines feasible long-term, given that such two-line genetic mixes don’t occur in nature?
    \item What are the interpretations of upward vs. downward contagion? 
\end{itemize}

\begin{thebibliography}{99}

\bibitem{ulrich18} Y. Ulrich, J. Saragosti, C. K. Tokita, C. E. Tarnita, D. J. C. Kronauer, ``Fitness benefits and emergent division of labour at the onset of group living,'' \textit{Nature}, vol. 560, pp. 635-638, Aug. 2018.

\bibitem{gautrais02} J. Gautrais, G. Theraulaz, J. L. Deneubourg, C. Anderson, ``Emergent polyethism as a consequence of increased colony size in insect societies,'' \textit{Journal of Theoretical Biology}, vol. 215, pp. 363–373, Apr. 2002.

\end{thebibliography}

\begin{appendices}

\section{Analytical model details} \label{sec:appendix}

\subsection{Model} \label{sec:model}

To consider the dynamics of division of labor in mixed colonies, we extend the fixed threshold model in \cite{ulrich18} to incorporate ants of two genetic lines, \A\ and \B. 

The model considers $n$ individuals and $m$ tasks. 
Let $f$ and $1-f$ be the fractions of genetic line \A\ and \B\ individuals in the colony, respectively.
Each task $j$ has an associated stimulus, $s_{j,t}$, at every time $t$, indicating the group-level demand for that task. We model the change in stimulus over discrete time as
\begin{equation}
    s_{j,t+1} - s_{j,t}  = \delta_j - \frac{\alpha_j^{\A} n_{j,t}^{\A} + \alpha_j^{\B} n_{j,t}^{\B} }{n}, \label{eq:stim}
\end{equation}
where $\delta_j$ is the task-specific demand rate; $\alpha_j^{\A}$ and $\alpha_j^{\B}$ are the task-specific performance efficiencies of lines \A\ and \B, respectively; and $n_{j,t}^{\A}$ and $n_{j,t}^{\B}$ are the numbers of line \A\ and line \B\ individuals performing task $j$ and time $t$, respectively.

At each time step, inactive individuals are exposed to the task stimuli randomly until they  either begin performing a task or have encountered all stimuli without landing on a task. Similar to \cite{ulrich18}, our computational model draws individual $i$'s internal threshold $\theta_{ij}$ for task $j$ from a normal distribution with mean $\mu_j$ and normalized standard deviation $\sigma_j$ (each of which can be line- and/or task-specific). To gain analytical insight into the model, we make the simplifying assumption that $\sigma_j = 0$ for all tasks.
In other words, \textit{we assume that the line- and task-specific thresholds are given by the constant parameters, $\mu_j^{\A}$ and $\mu_j^{\B}$.}
With this assumption, the probabilities $P_{ij,t}^{\A}$ and $P_{ij,t}^{\B}$ that inactive individuals $i$ of lines \A\ and \B\ begin to perform task $j$ at time $t$ can be written respectively as
\begin{equation}
    P_{j,t}^{\A} (s_{j,t}) = \frac{s_{j,t}^\eta}{s_{j,t}^\eta + {(\mu_j^{\A})}^\eta}, \quad P_{j,t}^{\B} (s_{j,t}) = \frac{s_{j,t}^\eta}{s_{j,t}^\eta +{(\mu_j^{\B})}^\eta}.
\end{equation}
The parameter $\eta$ governs the steepness of the response threshold function as in the original model.

Finally, at time $t$, active individuals quit their tasks with a constant quit probability $\tau$. In the case of two tasks ($m = 2$), the numbers of line \A\ and line \B\ individuals working on task $j$ change according to the following:
\begin{align}
    n_{j,t+1}^{\A} - n_{j,t}^{\A}  & =  \frac{1}{2} \bigg[ P_{j,t}^{\A}(s_{j,t}) + (1 - P_{j',t}^{\A}(s_{j',t}))P_{j,t}^{\A}(s_{j,t})\bigg] \bigg( fn
 - (n_{j,t}^{\A} + n_{j',t}^{\A}) \bigg) - \tau n_{j,t}^{\A} \nonumber\\
    n_{j,t+1}^{\B} - n_{j,t}^{\B}  & = \frac{1}{2} \bigg[ P_{j,t}^{\B}(s_{j,t}) + (1 - P_{j',t}^{\B}(s_{j',t}))P_{j,t}^{\B}(s_{j,t})\bigg] \bigg( (1-f)n - (n_{j,t}^{\B} + n_{j',t}^{\B}) \bigg) - \tau n_{j,t}^{\B} \label{eq:active}
\end{align}
for $(j,j')=(1,2), (2,1)$. {\color{black}The sums in square brackets capture the possible ways in which individuals can initiate task $j$: they can either encounter the stimulus for task $j$ immediately and begin performing that task, or they can first encounter the stimulus for the other task $j'$, decide not to perform that task, subsequently encounter the stimulus for task $j$, and begin performing task $j$.}

The full model with two tasks ($m=2$) consists of the six equations describing changes in $s_1, s_2$~\eqref{eq:stim} and $ n_1^{\A}, n_1^{\B}, n_2^{\A}, n_2^{\B}$~\eqref{eq:active}. 
% Exclude for now
% Given initial conditions 
% % $s_{1,0} \geq 0$, $s_{2,0} \geq 0$, 
% $0\leq n_{1,0}^{\A}+n_{2,0}^{\A} \leq fn$ and $0\leq n_{1,0}^{\B}+ n_{2,0}^{\B} \leq (1-f)n$,
% the dynamics will satisfy the constraints $0\leq n_{1,t}^{\A}+n_{2,t}^{\A} \leq fn$ and $0\leq n_{1,t}^{\B}+ n_{2,t}^{\B} \leq (1-f)n$ for all $t \geqq 0$.

\subsection{Steady-state predictions}
To understand how well the analtyical model captures the simulation results, we investigate the steady-state predictions of the analytical model for two tasks ($m=2$).

\subsubsection{A necessary condition for the existence of a biologically plausible equilibrium} \label{sec:maxactivity}

In our computational model, the individuals essentially have a latency period of one time step between when they quit a task and recommence working. This means that, on average, only a fraction of the colony can be working at any given time.

To find this maximal activity level, let $X_t = (n_{1,t}^{\A}+ n_{1,t}^{\B}+ n_{2,t}^{\A}+n_{2,t}^{\B})/n$ be the fraction of active individuals in a colony at time $t$. Note that $0\leq X_t\leq 1$. At time $t+1$, a fraction $\tau X_t$ of the colony is expected to be inactive. Therefore, at time $t+1$,
\begin{equation}
    \mathrm{(fraction\ active)} + \mathrm{(fraction\ inactive)}  = X_{t+1} + \tau X_t \leq 1.
\end{equation}
At steady state, $X_{t+1} = X_{t} = X^*$. By substitution, we obtain a necessary condition for a biologically plausible steady state to exist:
\begin{equation}
    X^* \leq \frac{1}{1+\tau}. \label{eq:maxactivity}
\end{equation}
For example, for $\tau = 0.2$ used in the simulations below, at most $83.33$\% of the individuals in a colony can be active at steady state.
A similar condition for the theoretically possible maximum activity level at steady state has been noted in \cite{gautrais02}.

\subsubsection{Pure colonies: 2 tasks, 1 line} \label{sec:pure}
For pure (\A-only) colonies, we set $f = 1$ and $n_{j,t}^{\B} = 0$ for all time $t$. Setting Eq.~\eqref{eq:stim} to zero, we obtain
\begin{equation}
    \frac{n_j^{\A}}{n} = \frac{\delta_j}{\alpha_j^{\A}}
    \label{eq:ss1}
\end{equation}
as the fraction of \A\ ants performing task $j$ at steady state.

Notably, the steady-state values of $n_j^{\A}$ are \textit{independent} of the mean threshold ($\mu_j^{\A}$) or the quit probability ($\tau^{\A}$). This agrees with the results of our simulations, where varying $\mu$ and $\tau$ by line did not lead to differing mean task performance levels in the pure colonies.

According to condition \eqref{eq:maxactivity}, this steady state is biologically possible only if
\begin{equation}
    (X^* =)\ \frac{n_1^{\A}}{n} + \frac{n_2^{\A}}{n} = \frac{\delta_1}{\alpha_1^{\A}} + \frac{\delta_2}{\alpha_2^{\A} }\leq \frac{1}{1+\tau} \label{eq:cond1}.
\end{equation}
If this condition is not met, then we would expect the stimuli to continue growing (i.e., the system will not reach a steady state).


\subsubsection{Mixed colonies: 2 tasks, 2 lines, 50-50 mixes} \label{sec:5050}

We now consider the 50-50 mixes ($f=0.5$), i.e., mixed colonies consisting of an equal number of \A\ and \B\ ants.
Here we assume that the mean thresholds and the quit probabilities are identical for both tasks and lines ($\mu_1^{\A} = \mu_2^{\A} = \mu_1^{\B} = \mu_2^{\B}$ and $\tau^{\A} = \tau^{\B}$)\footnote{The parameters $\mu$ and $\tau$ do not explicitly appear in \eqref{eq:ss2} when we assume that the mean thresholds are identical for all individuals and both tasks. However, based on the Eqs.~\eqref{eq:active}, we expect the general form of steady state fractions of active individuals to be explicit functions of $\mu_j^{\A}$ and $\mu_j^{\B}$ as well as $\tau^{\A}$ and $\tau^{\B}$. The steady states can be computed numerically for the case when these parameters vary, but the analytical expression is too complicated to write down.}, as we have done in the simulations with varied $\delta$ and $\alpha$ values. Setting Eqs.~\eqref{eq:stim} and \eqref{eq:active} equal to zero, we find the steady-state numbers of individuals performing task $j$ as
\begin{equation}
     n_j^{\A} =  n_j^{\B} = n\bigg(\frac{\delta_j}{\alpha_j^{\A} + \alpha_j^{\B}}\bigg). \label{eq:ss2a}
\end{equation}
Alternatively, as fractions of each genetic type of individuals,
\begin{equation}
     \frac{n_j^{\A}}{(n/2)} =  \frac{n_j^{\B}}{(n/2)} = \frac{2\delta_j}{\alpha_j^{\A} + \alpha_j^{\B}}. \label{eq:ss2}
\end{equation}
Applying condition \eqref{eq:maxactivity}, this state state exists only if
\begin{equation}
     (X^*=)\ \sum_{j=1}^2 \frac{n_j^{\A}}{n} + \frac{n_j^{\B}}{n} 
     = \sum_{j=1}^2 \frac{2\delta_j}{\alpha_j^{\A} + \alpha_j^{\B}}
     \leq \frac{1}{1+\tau}.
     \label{eq:cond2}
\end{equation}
Again, if this condition is not met, then we would expect the stimuli to continue growing over time and for the ants to be working at max capacity.

\subsubsection{Mixed colonies: 2 tasks, 2 lines, non-50-50 mixes}

We now generalize to the case in which a fraction $f$ of individuals in a mixed colony are of genetic line \A. In the simplified case where $\mu_1^{\A} = \mu_2^{\A} = \mu_1^{\B} = \mu_2^{\B}$ and $\tau^{\A} = \tau^{\B}$, the steady-state fractions of individuals performing task $j$ are
\begin{equation}
     n_j^{\A} =  \frac{fn\delta_j}{f\alpha_j^{\A} + (1-f)\alpha_j^{\B}}, 
     \quad
     n_j^{\B} =  \frac{(1-f)n\delta_j}{f\alpha_j^{\A} + (1-f)\alpha_j^{\B}}, 
     \label{eq:ss3a}
\end{equation}
or, as fractions of the individuals of each genetic type (recall that there are $fn$ individuals of type \A\ and $(1-f)n$ individuals of type \B),
\begin{equation}
     \frac{n_j^{\A}}{fn} =  \frac{n_j^{\B}}{(1-f)n} = \frac{\delta_j}{f\alpha_j^{\A} + (1-f)\alpha_j^{\B}} \ \bigg(= \frac{n_j^{\A} + n_j^{\B}}{n}\bigg). \label{eq:ss3}
\end{equation}
The last equality highlights the fact that the fraction of individuals \textit{of each type} performing task $j$ is identical to the fraction \textit{of the whole colony} performing that task. As expected, the expressions \eqref{eq:ss3} reduce to \eqref{eq:ss2} when $f=0.5$ (50-50 mixes) and to \eqref{eq:ss1} when $f=1$ (pure \A\ colonies). Again, we would expect to see this equilibrium only when condition \eqref{eq:maxactivity} is satisfied. 

From these results, we would expect the steady-state fraction of task $j$ performance frequency (which corresponds to \eqref{eq:ss3}) to depend \textit{nonlinearly} on the fraction of \A\ individuals ($f$). 
\subsubsection{Mixed colonies: 2 tasks, 2 lines, 50-50 mixes, symmetric mean thresholds}

So far we have assumed that the mean thresholds $\mu_{j}^{\A}$ and $\mu_{j}^{\B}$ were identically for both lines and tasks ($\mu_{1}^{\A} = \mu_{2}^{\A} = \mu_{1}^{\B} = \mu_{2}^{\B}$). While the system of equations can be solved numerically even when we vary $\mu$, the steady-state expressions quickly become too complicated to write down.

In the following special case, however, we can express the equilibrium values exactly. Assume that 
\begin{itemize}
    \item 1) the task efficiencies are the same for both lines and tasks ($\alpha_{1}^{\A} = \alpha_{2}^{\A} = \alpha_{1}^{\B} = \alpha_{2}^{\B} = \alpha$);
    \item 2) the quit probabilities are the same for both tasks ($\tau^{\A} = \tau^{\B} = \tau$); and
    \item 3) the task demand rates are the same for both tasks ($\delta_1 = \delta_2  = \delta$); 4) the task thresholds are ``symmetric'': $\mu_{1}^{\A} = \mu_{2}^{\B} = a$ and $\mu_{2}^{\A} = \mu_{1}^{\B} = b$. 
\end{itemize}

The symmetry of the parameters means that the dynamics of the stimuli are identical. Therefore, at steady state, the stimulus levels should be equal ($s_1 = s_2 = s$). Moreover, the number of \A\ ants performing task 1 should be identical to that of \B\ ants performing task 2 ($n_{1}^{\A} = n_{2}^{\B}$) and the number of \B\ ants performing task 1 should be identical to that of \A\ ants performing task 2 ($n_{1}^{\B} = n_{2}^{\A}$).

Taking advantage of this symmetry, we can write down the steady-state stimulus level $s$ $(=s_1 = s_2)$ as
\begin{equation}
    s^* = \bigg[\frac{1}{2} \bigg( -({a}^\eta + {b}^\eta) 
    \pm \sqrt{
    ({a}^\eta + {b}^\eta)^2 + ({a}^\eta {b}^\eta)\cdot \frac{8\delta \tau}{\alpha - 2\delta (1+\tau)}
    } \bigg)\bigg]^\frac{1}{\eta}.
\end{equation}
The corresponding steady-state fractions of \A\ and \B\ ants performing tasks 1 and 2 are
\begin{equation}
    \frac{n_1^{\A}}{(n/2)} =  \frac{n_2^{\B}}{(n/2)} = \frac{1}{\tau} \bigg(\frac{(s^*)^\eta}{(s^*)^\eta + {a}^\eta} \bigg)\bigg[2 - \frac{(s^*)^\eta}{(s^*)^\eta + {b}^\eta}  \bigg] \bigg(\frac{1}{2} - \frac{\delta}{\alpha} \bigg)
    \label{eq:sol4}
\end{equation}
\begin{equation}
    \frac{n_2^{\A}}{(n/2)} =  \frac{n_1^{\B}}{(n/2)}  = \frac{1}{\tau} \bigg(\frac{(s^*)^\eta}{(s^*)^\eta + {b}^\eta} \bigg)\bigg[2 - \frac{(s^*)^\eta}{(s^*)^\eta + {a}^\eta}  \bigg] \bigg(\frac{1}{2} - \frac{\delta}{\alpha} \bigg)
    \label{eq:sol5}
\end{equation}
When $a = b$ (i.e., when all $\mu$'s are the same), \eqref{eq:sol4} and \eqref{eq:sol5} reduce to the steady-state values predicted in \eqref{eq:ss2}.


\subsection{Downward vs. upward contagion}

One question that was raised during our previous discussion was whether varying both the task efficiency $\alpha$ and the demand rate $\delta$ would capture the difference in the directions of contagion (downward or upward) when the larvae differed. In Section~\ref{sec:varyalphadelta}, we presented simulation results that suggest that upward contagion only occurs when some of the ants are inefficient and therefore the colony cannot maintain the stimuli at a steady level. Here we show analytically that, \textit{if the system reaches its steady state (i.e., condition \eqref{eq:maxactivity} is satisfied), then varying \textbf{only} the demand rate ($\delta$) and task efficiency ($\alpha$) can result in a downward contagion but not an upward contagion.}

If we only vary $\delta$ and $\alpha$, then the mean thresholds and the quit probabilities are identical across tasks and lines. So we can directly apply the steady-state fractions of active individuals computed in Eqs.~\eqref{eq:ss1} and \eqref{eq:ss2}. The behavioral contagion is ``downward'' if
\begin{equation}
    \frac{1}{2} \bigg( \frac{\delta_j}{\alpha_j^{\A}} + \frac{\delta_j}{\alpha_j^{\B}} \bigg) > \frac{2\delta_j}{\alpha_j^{\A} + \alpha_j^{\B}} \label{eq:down}
\end{equation}
and ``upward'' if the inequality is reversed (the schematic below should help).
\begin{figure}[H]
    \centering
    \includegraphics[width=0.9\linewidth]{doc/schematic_contagion.pdf}
    \caption{A schematic representing the two contagion patterns of our interest. The task $j$ performance (\%) values for the mixed colonies assume that the mean thresholds and the quit probabilities are identical for both lines and both tasks ($\mu_1^{\A} = \mu_2^{\A} = \mu_1^{\B} = \mu_2^{\B}$ and $\tau^{\A} = \tau^{\B}$).}
    \label{fig:schematic}
\end{figure}
\noindent By manipulating the inequality \eqref{eq:down}, however, we see that the LHS is always at least as large as the RHS:
\begin{align}
    \frac{1}{2} \bigg( \frac{\delta_j}{\alpha_j^{\A}} + \frac{\delta_j}{\alpha_j^{\B}} \bigg) - \frac{2\delta_j}{\alpha_j^{\A} + \alpha_j^{\B}} 
    % & = \frac{\delta_j}{2} \bigg( \frac{1}{\alpha_j^{\A}} + \frac{1}{\alpha_j^{\B}} - \frac{4}{\alpha_j^{\A} + \alpha_j^{\B}} \bigg) \nonumber\\
    % & = \frac{\delta_j}{2} \Bigg( 
    % \frac{\alpha_j^{\B} (\alpha_j^{\A} + \alpha_j^{\B}) + \alpha_j^{\A} (\alpha_j^{\A} + \alpha_j^{\B}) - 4\alpha_j^{\A}\alpha_j^{\B}}{\alpha_j^{\A}\alpha_j^{\B}(\alpha_j^{\A} + \alpha_j^{\B})} \Bigg) \nonumber\\
    % & = \frac{\delta_j}{2} \Bigg( 
    % \frac{ (\alpha_j^{\A})^2 + (\alpha_j^{\B})^2 - 2\alpha_j^{\A}\alpha_j^{\B}}{\alpha_j^{\A}\alpha_j^{\B}(\alpha_j^{\A} + \alpha_j^{\B})} \Bigg) \nonumber\\
    & = \frac{\delta_j}{2} \Bigg( 
    \frac{ (\alpha_j^{\A} - \alpha_j^{\B})^2 }{\alpha_j^{\A}\alpha_j^{\B}(\alpha_j^{\A} + \alpha_j^{\B})} \Bigg) \geq 0
\end{align}
The equality holds if and only if $\alpha_j^{\A} = \alpha_j^{\B}$, in which case the ants are indistinguishable with respect to task~$j$ (the three lines take the same value for task $j$). If $\alpha_j^{\A}
\neq \alpha_j^{\B}$, then only downward contagion is possible under our assumptions.

Upward contagion is still possible if we relax some of our assumptions. For example,
\begin{itemize}
    \item when the mean threshold ($\mu$) is varied in addition to the task efficiency ($\alpha$), or
    \item when some ants are too inefficient to maintain the stimuli at constant levels (i.e., reach steady state).
\end{itemize}

\end{appendices}

\end{document}
%%%%%%%%%%%%%%%%%%%%%%%%%%%

